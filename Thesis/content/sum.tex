\chapter{Összegzés}
\label{ch:sum}


Készítettem egy kétdimenziós alakzat szerkesztő programot, amely kiszámolja a pontos előjeles távolságfüggvényét. A program képes személyre szabhatóan, a szerkesztés közben valós időben vizualizálni az előjeles távolságot. A távolságfüggvényt meghatározó algoritmus a síkot felosztja Voronoj-cellákra, mellyel a távolságfüggvény nagyon gyorsan kiértékelhető a GPU-n. Az algoritmus kimenetét az alkalmazás képes három dimenzióban megjeleníteni.


\subsection{Továbbfejlesztési lehetőségek}

Egy mondás szerint egy szoftver tervezését és fejlesztését csak abbahagyni lehet, befejezni nem. Szeretnék felsorolni néhány lehetőséget:

\begin{itemize}
    \item Szerkesztés engedélyezése a háromdimenziós nézetben
    \item Köztes állapotok módosítása a szerkesztőben
    \item Algoritmus továbbfejlesztése, hogy önmetsző alakzatokra is működjön
    \item További vizualizációs paraméterek bevezetése
\end{itemize}
